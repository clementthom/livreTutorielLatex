%on peut commenter avec le symbole pourcent %
\documentclass[12pt]{lettre} %type de document : lettre (peut être article ou autre)

\usepackage[utf8]{inputenc}%on installe les paquets nécessaires
\usepackage[T1]{fontenc}
\usepackage{lmodern}
\usepackage{eurosym}
\usepackage[french]{babel}
\usepackage{numprint}

\begin{document}
%un motif lettre est appliqué avec les champs correspondants
\begin{letter}{Service des réclamations}
\name{Jean Râleur}
\address{Jean Râleur\\4, rue du Bac à sable\\80886 Sassone- le-Creux}
\lieu{Sassone-le-Creux}
\telephone{01 02 03 04 05}
\email{jean.raleur@fai.fr}
\nofax

\def\concname{Objet :~} % On définit ici la commande 'objet'
\conc{rétractation}
\opening{Madame, Monsieur,}

Vous m'avez démarché la semaine dernière
pour me proposer l'édition de luxe de l'encyclopédie Wikipédia,
pour la somme de \numprint{5000}~\euro. %on écrit des numéros et le symbole euro
Conformément à la loi m'accordant un délai de rétractation de 7~jours, %le ~ indique la fin du nombre
je renonce à mon achat et demande le remboursement de la somme versée.


\closing{Je vous prie d'agréer,
Madame, Monsieur,
mes salutations distinguées.}

\end{letter}

\end{document}