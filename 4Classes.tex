\documentclass[a4paper, 12pt]{article}
\usepackage[utf8]{inputenc} 
\usepackage[T1]{fontenc}
\usepackage{lmodern}
\usepackage{graphicx}
\usepackage[french]{babel}

\begin{document}
La première chose à écrire dans un document \LaTeX~ est la classe du document.
On la définit de cette façon : \textbackslash documentclass[options]\{article\} .

Voici les différents types de documents :
\begin{itemize}
\item article : pour des articles destinés à la publication et ne contenant que quelques pages.
\item report : pour des documents un peu plus longs contenants plusieurs chapitres, comme des mémoires de thèse.
\item book : pour de véritables livres, de plusieurs centaines de pages.
\item slides : pour faire des présentations sur transparents.

\item beamer : pour faire des présentations en utilisant l'extension beamer (très bien d'après le site).
\item lettre : pour faire des lettres au format français (classe écrite par l'Observatoire de Genève).
\item memoir : pour écrire des mémoires, par exemple de fin d'étude.
\end{itemize}

Il en existe s\^urement d'autres, mais il faudrait faire une recherche approfondie afin de les lister ici.

Classes {\em book} et {\em report}, voir le document 4 1

\end{document}