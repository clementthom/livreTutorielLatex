\documentclass[a4paper,  10pt]{book}
\usepackage[utf8]{inputenc}
\usepackage[T1]{fontenc}
\usepackage{lmodern}
\usepackage{graphicx}
\usepackage[french]{babel}
\usepackage{cite}


\begin{document}

\frontmatter %déclare la couverture, que pour les livres

\title{La structuration des documents en LaTeX}
\author{Clément THOMAS}
\date{Le 18 juin 2024}

\mainmatter %on est dans le corps du document

\part{Les commandes LaTex pour structurer un texte}
\chapter*{Introduction}
Pour faire un chapitre (ou quoi que ce soit d'ailleurs) sans précisions (pas de \em chapitre 1\em   ou similaire), on rajoute une * après chapter.

\chapter{La structure d'un document}
\section{Listes des instructions}
Voici la hérarchie d'un texte : 
\begin{itemize}

\item 1 : Partie : la commande est \textbackslash part\{nom de la partie\} .
\item 2 : Chapitre (n'existe pas avec \em article\em) : la commande est \textbackslash chapter\{nom du chapitre\}
\item 3 : Section : la commande est \textbackslash section\{nom de la section\}
\item 4 : Sous-section : la commande est \textbackslash subsection\{nom de la sous-section\}
\item 5 : Sous-sous-section : la commande est \textbackslash subsubsection\{nom de la sous-sous-section\}
\item 6 : Paragraphe : la coommande est \textbackslash paragraph\{titre du paragraphe\}
\item 7 : Sous-paragraphe la commande est \textbackslash subparagraph\{titre du sous-paragraphe\}

\end{itemize}
\section{Notes importantes}
\begin{itemize}

\item Les chapitres n'existent que dans les livres, ce qui parait logique : pas de chapitres dans un article.
\item Par convention françaisej, la table des matières est placée à la fin du document (écrite\em\textbackslash tableofcontents\em).
\item Pour les classe \em article \em et \em report \em , on a un duo supplémentaire pour faire un abstract. Il s'écrit de la même manière que par exemple \em itemize \em : \textbackslash begin\{abstract\} blablabla \textbackslash end\{abstract\} .
\item Il faut compiler deux fois pour que la table des matières se mette à jour.
\item Pour écrire en italique (emphase), il faut employer la syntaxe : \textbackslash emph\{texte\}.

\end{itemize}

\nopagebreak

\chapter{La structure du texte}
\section{Le saut de page}
Insérer un saut de page est assez simple. En effet il est gérer automatiquement par \LaTeX . En revanche, on peut lui indiquer nos préférences, que ce soit pour sauter une page avec \em \textbackslash pagebreak\em (insérer saut de page) , ou au contraire pour ne pas sauter une page avec\em \textbackslash nopagebreak\em .

\section{Listes, imbrications et descriptions}
\subsection{Les listes numérotées}

La liste numérotée utilise une syntaxe type \emph{\textbackslash begin\{\}}, avec entre les accolades \emph{enumerate} et des \emph{item}.
Voici un exemple :
\begin{enumerate}

\item premier item
\item deuxième item

\end{enumerate}	

\subsection{Les listes non numérotées}

C'est la même chose pour les listes non numérotées, mais on entre \emph{itemize} au lieu de \emph{enumerate}.
\begin{itemize}

\item première partie
\item deuxième partie

\end{itemize}

\subsection{Les descriptions}

L'environnement \emph{description} permet d'associer une définition à un terme (un environnement commence par \textbackslash begin\{\} et se termine par \textbackslash end\{\}.  
En plus d'écrire \emph{item}, on ajoute entre crochets le nom du terme qu'on souhaite associer à une définition. Le résultat est ceci :
\begin{description}

\item[die Entdeckung] : La découverte
\item[das Gehirn] : le cerveau

\end{description}



\subsection{Les imbrications}

Juste pour dire qu'on peut inclure des listes dans des listes.

\section{Notes et sources dans le document}
\subsection{Notes de bas de page}

Pour créer une note de bas de page \footnote{Comme ici.}, on entre la commande \emph{\textbackslash footnote} devant le mot à annoter. 

\subsection{Références}

On veut parfois faire référence à quelque chose déjà mentionné dans le texte. Pour cela, on utilise deux commandes : 
\begin{itemize}

\item On écrit \emph{\textbackslash label\{étiquette\}} à l'endroit que l'on prend comme référence
\item On écrit \emph{\textbackslash ref\{étiquette\}} pour obtenir la position de l'étiquette dans le document (voir exemple)
\item On écrit \emph{\textbackslash pageref\{étiquette\}} pour obtenir le numéro de la page où est située l'étiquette.  

\end{itemize}

À noter que si l'étiquette n'a pas été déclarée et a été appelée, ou qu'elle est déclarer à plusieurs reprises, une erreur sera retournée.

Voici un exemple d'utilisation d'étiquette :

Les points forts de \LaTeX{} sont~:\label{points-forts}
[…]
Comme relevé précédemment (section~\ref{points-forts} p.~\pageref{points-forts}), […]

\chapter{Citation de documents externes}
\section{Fichier de bibliographique}

On fait appel à un fichier de référence, qui contient les informations des livres/articles concernés. Voici un éxemple de la rédaction d'informations d'un livre :

\% ***** livres *****

@book\{VER1875,

   author="Verne, Jules",

   title="Michel \{Strogoff\}",

   year="1875",

   publisher="Le livre de poche"
\}

A noter qu'on créer un fichier spécial (un répertoire) où on dispose toutes les références bibliographiques. Ce fichier est un fichier de type \emph{.bib}, qu'on a pas besoin de lier avec le fichier \emph{.tex}. 

\section{Référencer des documents}

Il y a deux façons de citer un ouvrage 
\begin{itemize}

\item \textbackslash \emph{cite\{VER1875\}}, passé à côté de la citation
\item \textbackslash \emph{nocite\{VER1875\}}, lorsqu'on ne cite pas explicitement le document externe dans le texte. A noter qu'il est préférable de placer ces \oe uvres juste avant \emph{\textbackslash backmatter}

\end{itemize}

Voici un exemple :

Ainsi, Jules Verne faisait dire à Wassili Fédor \cite{VER1875} : (bon, là ça ne marche pas).


\backmatter % on est au niveau de la table des matières et des index

\bibliographystyle{plain-fr}
\bibliography{mabiblio}

\tableofcontents

\end{document}

