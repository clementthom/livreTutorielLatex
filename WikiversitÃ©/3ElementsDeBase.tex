\documentclass[a4paper, 12pt]{article}
\usepackage[utf8]{inputenc} 
\usepackage[T1]{fontenc}
\usepackage{lmodern}
\usepackage{graphicx}
\usepackage[french]{babel}

\begin{document}
\emph{texte} %italique
Salut princesse!
{\em T'as bien dormi ?} %italique

%Les caractères { } % # $ ^ ~ & _ \ sont réservés

On peut de même écrire les \{caractères spéciaux\} avec un antislash \textbackslash.
Pour faire un \textasciicircum, il faut écrire \textbackslash textasciicircum.
Pour faire un \textbackslash, il faut écrire \textbackslash textbackslash.
Pour faire un tilde \textasciitilde, il faut écrire \textbackslash textasciitilde.

%un saut de ligne dans le .ext effectue un retour à la ligne dans le pdf
%plusieurs sauts de lignes ne créent pas de saut de ligne dans le document
Voici des exemples pour placer des caractères spéciaux issus de lettres : 

\'e (\'ecrit \textbackslash 'e)

\`e (\'ecrit \textbackslash `e, altgr 7)

\^\i  (\'ecrit \textbackslash  \^{}\textbackslash  i) % les {} permettent de fermer l'instruction ^

D'autres caractères spéciaux (cédille, tréma, \oe , ...) peuvent être trouvé en suivant ce lien :

https:\textbackslash\textbackslash fr.wikibooks.org\textbackslash wiki\textbackslash LaTeX\textbackslash \'Eléments\_de\_base
% ne pas mettre d'espace après les \_

Il faudra supprimer l'espace apparu avant le :, je n'ai pas trouvé de moyen pour le faire disparaître.
On peut aussi créer plusieurs types d'espaces (avec chacun une longueur différente).

\end{document}