\documentclass[a4paper, 8pt]{article}

\usepackage[utf8]{inputenc}
\usepackage[T1]{fontenc}
\usepackage{lmodern}
\usepackage{graphicx}
\usepackage[french]{babel}
\usepackage{cite}
\usepackage{array}

\usepackage[utf8]{inputenc}
\usepackage[T1]{fontenc}
\usepackage{lmodern}
\usepackage{graphicx}
\usepackage[french]{babel}
\usepackage{cite}
\usepackage{array}

\begin{document}

\section{Faire un tableau}
\subsection{Commandes de bases pour créer un tableau}
Pour créer un tableau, on utilise l'environnement \textit{\textbackslash textit\{tabular\}\{lll\}}, où \texttt{lll} représente le nombre de colonnes (ce sont des L minuscules). //
Voici un exemple de tableau créé avec cette méthode : \\ \\

\begin{tabular}{lll}
   1.1 & 1.2 & 1.3 \\
   2.1 & 2.2 & 2.3 \\
\end{tabular}

Il y a plusieurs façons de placer le contenu d'une case :
\begin{itemize}

\item On utilise "l" pour mettre le contenu d'une case à gauche
\item On utilise "r" pour mettre le contenu à droite
\item On utilise "c" pour mettre le contenu au centre
\item Pour un alignement à gauche avec marge, on utilise \textit{p\{largeur\}}, où peut mettre par exemple 3cm à la place de \textit{largeur}\\ \\

\end{itemize} 

\textsc{attention !} : Pour faire un tableau avec des traits, il faut rajouter des traits "|" entre les élements suivant le \textit{\{tabular\}}. Voici un exemple du tableau précédent avec \textit{\textbackslash textit\{tabular\}\{ l | l | l \}} : \\

\begin{tabular}{l | l | l }
	\hline 
	colonne1 & colonne 2 & colonne 3 \\
	\hline \hline
   1.1 & 1.2 & 1.3 \\
   2.1 & 2.2 & 2.3 \\
	\hline
\end{tabular}\\ \\

On utilise \textit{\textbackslash hline} au début de la première ligne, avant le contenu du tableau et à sa fin, juste avant le \textit{\textbackslash end} : cela créé les traits horizontaux. On peut en mettre deux, comme dans l'exemple, pour avoir une ligne double.

\subsection{Fusionner des colonnes et des lignes}
Pour fusionner des colonnes, on utilise la commande \\
\textit{\textbackslash multicolumn\{nombre de colonnes fusionnées\}\{alignement comme celui de tabular\}\{texte des deux colonnes à la première ligne\}}. Exemple :\\ \\

\begin{tabular}{|l|c|r|}
   \hline
   colonne 1 & \multicolumn{2}{c|}{colonnes 2 et 3} \\
   \hline
   1.1 & 1.2 & 1.3 \\
   \hline
   2.1 & 2.2 & 2.3 \\
   \hline
\end{tabular}\\ \\

La méthode pour fusionner des lignes est un peu différente. On utilise la commande \textit{\textbackslash cline\{2-3\} \&} pour sélectionner les lignes \textsc{à ne pas fusionner}, ici la ligne 2 et 3.
Voici un exemple de rendu : \\ \\

\begin{tabular}{|l|c|r|} 
   \hline
    colonne 1 & \multicolumn{2}{c|}{colonnes 2 et 3} \\
    \hline
    1.1 & 1.2 & 1.3 \\
    \cline{2-3} 
        & 2.2 & 2.3 \\
    \hline
\end{tabular}

\subsection{Tableau flottant et positionnement}
Au lieu d'avoir un tableau placé par défaut par \LaTeX à gauche, on peut créer un tableau flottant et laisser le typographe le placer au mieux selon les directives exprimées. \\
On le place tout d'abord dans un environnement \textit{\{table\}[position]}. Il y a 4 types de position :
\begin{itemize}

\item \textit{h} pour qu'il soit à côté du texte le précédant (here)
\item \textit{t} pour le mettre en haut de la page
\item \textit{b} pour le mettre en bas de la page
\item \textit{p} pour le mettre dans une page ne regroupant que des flottants (regroupement de figures et de tableaux)

\end{itemize}
\`A noter que le tableau est numéroté, ce qui permet de dresser un index des tableaux. \\ \\
Si on veut donner un titre et placer une étiquette permettant de faire référence au tableau, on peut utiliser la commande \textit{\textbackslash caption\{label\{etiquette\} titre\}}.
Pour centrer le tableau sur la page, il est préférable d'utiliser la commande \textit{\textbackslash centering}.\\ \\
On place ces deux commandes entre le \textit{table} et le \textit{tabular}. Démonstration : \\ 

\begin{table}[h]
   \caption{\label{} Tableau sans titre}
   \centering
   \begin{tabular}{l | l | l }\\
	 \hline 
	colonne1 & colonne 2 & colonne 3 \\
	\hline \hline
   1.1 & 1.2 & 1.3 \\
   2.1 & 2.2 & 2.3 \\
	\hline
\end{tabular}
\end{table}

\textsc{notes :}on peut utiliser la commande \textit{\textbackslash clearpage} pour changer de page et afficher toutes les figures en attente sur celle-ci. On peut aussi utiliser la commande \textit{\textbackslash cleardoublepage} pour faire la même chose, mais sur une page impaire {utile lorsqu'on veut les figures sur le côté recto de la feuille lors d'une impression). \\
On peut créér une table des matières des tableaux avec la commande \textit{\textbackslash listoftables}, mais aussi faire figurer cet index dans la table des matières générale avec l'extension \texttt{\textbackslash tocbibind}.

\listoftables

\end{document}